% ****************************************************************************************************
% classicthesis-config-LyX.tex
% Use it in the preamble of your ClassicThesis.lyx
% ****************************************************************************************************
% If you like the classicthesis, we would appreciate a postcard.
% André's address can be found in the file ClassicThesis.pdf. A collection
% of the postcards received so far is available online at
% http://postcards.miede.de
% ****************************************************************************************************


% ****************************************************************************************************
% 1. Configure classicthesis for your needs here. At some point you will probably want to set 
% "drafting" to false in order to deactivate the time-stamp on the bottom of the pages
% See ClassicThesis.pdf for more information.
% ****************************************************************************************************
\PassOptionsToPackage{
	drafting=false,      % print version information on the bottom of the pages
	tocaligned=false,   % the left column of the toc will be aligned (no indentation)
	dottedtoc=false,    % page numbers in ToC flushed right
	parts=true,         % use part division
	eulerchapternumbers=true, % use AMS Euler for chapter font (otherwise Palatino)
	linedheaders=false,       % chaper headers will have line above and beneath
	floatperchapter=true,     % numbering per chapter for all floats (i.e., Figure 1.1)
	eulermath=false,    % use awesome Euler fonts for mathematical formulae (only with pdfLaTeX)
	beramono=true,      % toggle a nice monospaced font (w/ bold)
	palatino=false,      % toggle the standard roman font, see end of this file for more suggestions
	style=classicthesis % classicthesis, arsclassica
}{classicthesis}




% ****************************************************************************************************
% 2. Personal data and user ad-hoc commands
% ****************************************************************************************************
\newcommand{\myTitle}{Analiza bioinformatyczna, strukturalna i~funkcjonalna międzybiałkowych miejsc wiązania jonów Zn(II)\xspace}
\newcommand{\mySubtitle}{Praca doktorska\xspace}
\newcommand{\myDegree}{Magister (mgr)\xspace}
\newcommand{\myName}{Józef Tran\xspace}
\newcommand{\myProf}{Artur Krężel\xspace}
% \newcommand{\myOtherProf}{Put name here\xspace}
% \newcommand{\mySupervisor}{Put name here\xspace}
\newcommand{\myFaculty}{Wydział Biotechnologii\xspace}
\newcommand{\myDepartment}{Zakład Chemii Biologicznej\xspace}
\newcommand{\myUni}{Uniwersytet Wrocławski\xspace}
\newcommand{\myLocation}{Wrocław\xspace}
\newcommand{\myTime}{Maj 2023\xspace}
\newcommand{\myVersion}{v1.3}

% There is a problem using ę and ą with Mendeley, ę was expressed as \c{e}, ą as \c{a}
\renewcommand{\c}{\k}
\DeclareUnicodeCharacter{0229}{\k{e}}
\DeclareUnicodeCharacter{0327}{0328}
\DeclareUnicodeCharacter{2032}{$\prime$}

% \DeclareUnicodeCharacter{0301}{◌́◌́}



\newcommand\MarginFig[4][width=\marginparwidth]{%
\marginpar{\includegraphics[#1]{#2}
\captionof{figure}{#3}
\label{#4}}
}

% ****************************************************************************************************


% ****************************************************************************************************
% 3. Loading some handy packages
% ****************************************************************************************************
\usepackage{csquotes} % biblatex depends on these

\usepackage{scrhack} % fix warnings when using KOMA with listings package
\usepackage{xspace} % to get the spacing after macros right
\usepackage{mparhack} % get marginpar right
\PassOptionsToPackage{printonlyused,smaller}{acronym}
\usepackage{acronym} % nice macros for handling all acronyms in the thesis
%\renewcommand{\bflabel}[1]{{#1}\hfill} % fix the list of acronyms --> no longer working
%\renewcommand*{\acsfont}[1]{\textsc{#1}} 
%\renewcommand*{\aclabelfont}[1]{\acsfont{#1}}
\def\bflabel#1{{\acsfont{#1}\hfill}}
\def\aclabelfont#1{\acsfont{#1}}

% Package to use Python code in the thesis 
\usepackage[gobble=auto]{pythontex}
\def \thesispath{"C:\string\Users\string\makro\string\Desktop\string\PhD"}
% 

\usepackage{newpxtext,newpxmath}
\useosf
\usepackage{textgreek}
\usepackage[version=4]{mhchem}



\hyphenation{mon-o-meric}
\hyphenation{en-donu-cle-olytic}
\hyphenation{Cho-rą-żew-ska}
\hyphenation{cha-ra-kte-ry-sty-ka}
\hyphenation{za-cho-wa-nie}

\babelprovide[transforms = oneletter.nobreak]{polish} 

%*****************************************************************************************************
% Colors
% ****************************************************************************************************

% ****************************************************************************************************
% Glossary
% ****************************************************************************************************
% \usepackage[acronym]{glossaries}
\usepackage[automake,nolist,nopostdot,sort=standard,acronym,]{glossaries}

% \setlength{\glsdescwidth}{\glsdescwidth-\glspagelistwidth}
\setlength{\glsdescwidth}{\glsdescwidth-6\tabcolsep}
% glossaries style
\newglossarystyle{no_indentation_long}{% 
	\setglossarystyle{long}
	\renewenvironment{theglossary}%
	{\begin{longtable}[l]{@{}p{4cm}p{\glsdescwidth}}}%
			{\end{longtable}}
}

\makeglossaries


\newglossaryentry{apig}{name={API},
	description={Interfejs programowania aplikacji (API) to zestaw
	zasad i~protokołów, które pozwalają dwóm lub większej ilości komponentów oprogramowania
	na komunikację między sobą.}
	}

\newglossaryentry{api}{type=\acronymtype, name={API},
	description={interfejs programowania aplikacji (ang. Application
			Programming Interface), \glsseeformat[Słownik terminów:]{apig}{}},
	first={interfejs programowania aplikacji  (API, ang. Application
	Programming Interface)\glsadd{apig}}
	}




\newglossaryentry{ppi}{type=\acronymtype, name={PPI},
	description={interakcja białko--białko (ang. protein--protein interaction)},
	first={interakcja białko-białko (PPI, ang. protein--protein interaction)},
	plural={PPIs},
	descriptionplural={interakcje białko--białko (ang. protein--protein interactions)},
	firstplural={interakcje białko--białko (\glsentryplural{ppi}, ang. protein--protein interactions)}
	}



\newglossaryentry{proteome}{name={proteom},
	description={Proteom to kompletny zestaw białek,
	które mogą być lub są wyrażane przez organizm.}
	}

\newglossaryentry{htsg}{name={HTS},
	description={Badania przesiewowe o wysokiej wydajności to zastosowanie
	robotyki, a~następnie przetwarzania danych do badania dużej
	liczby próbek.}
	}

\newglossaryentry{hts}{type=\acronymtype, name={HTS},
	description={badania przesiewowe o wysokiej wydajności  (ang. high-throughput screening, \glsseeformat[Słownik terminów:]{htsg}{})},
	first={badania przesiewowe o wysokiej wydajności (HTS, ang. high-throughput screening) \glsadd{htsg}},
	firstplural ={badania przesiewowe o wysokiej wydajności (HTS, ang. high-throughput screening))\glsadd{htsg}} 
	}

\newglossaryentry{interface}{name={interfejs},
	description={Wspólna powierzchnia pomiędzy dwoma lub wiekszą ilością komponentów (np. cząsteczkami).
	W przypadku białek powierzchnia oddziaływania składa się z~reszt aminokwasowych, które
	stykają się z~resztami innego partnera interakcji. Czasami
	interfejsy międzybiałkowe mogą zawierać dodatkowych partnerów interakcji, np. kofaktory białkowe
	i~ligandy.}
	}

\newglossaryentry{zppig}{name={ZPPI},
	description={Interakcja białko--białko z~udziałem Zn(II)
	 jest interakcją w której kluczową rolę odgrywa jon Zn(II).
	 W tej pracy ZPPI będzie odnosić się głównie do interfejsów
	 tworzonych poprzez kokoordynację jonu Zn(II) przez dwie lub
	 więcej makromolekuł.}
	}

\newglossaryentry{zppi}{type=\acronymtype, name={ZPPI},
	description={interakcja białko--białko z~udziałem Zn(II) (ang. Zn(II)-involved protein--protein interaction), \glsseeformat[Słownik terminów:]{zppig}{}},
	first={interakcja białko--białko z~udziałem Zn(II) (ZPPI, ang. Zn(II)-involved protein--protein interaction)\glsadd{zppig}},
	plural={ZPPIs},
	descriptionplural={interakcje białko--białko z~udziałem Zn(II) (ang. Zn(II)-involved protein--protein interactions)},
  firstplural={interakcje białko--białko z~udziałem Zn(II) (ang. Zn(II)-involved protein--protein interactions) (\glsentryplural{zppi})}
  }


\newglossaryentry{mppig}{name={MPPI},
description={Interakcja białko--białko z~udziałem metalu jest
 interakcją w której kluczową rolę odgrywa jon metalu.
 W tej pracy MPPI będzie odnosić się głównie do interfejsów
 tworzonych poprzez kokoordynację jonu metalu przez dwie lub
 więcej makromolekuł.}
}

\newglossaryentry{mppi}{type=\acronymtype, 
name={MPPI},
description={interakcja białko--białko z~udziałem metalu (ang. metal-involved protein--protein interaction), \glsseeformat[Słownik terminów:]{mppig}{}},
first={interakcja białko--białko z~udziałem metalu (MPPI, ang. metal-involved protein--protein interaction) (MPPI)\glsadd{mppig}},
plural={MPPIs},
descriptionplural={interakcje białko--białko z~udziałem metalu (ang. metal-involved protein--protein interactions)},
firstplural={interakcje białko--białko z~udziałem emtalu (ang. metal-involved protein--protein interactions)}
}





\newglossaryentry{cry}{type=\acronymtype,
	name={CRY1},
	description={białko kryptochrom 1 (ang. cryptochrome 1)},
	first={białko kryptochrom 1 (CRY1, ang. cryptochrome 1)},
}

\newglossaryentry{per}{type=\acronymtype,
	name={PER2},
	description={białko period 2},
	first={ białko period 2 (PER2)}
}


\newglossaryentry{freezn}{
	name={wolny Zn(II)},
	description={Frakcja jonów cynku, która jest labilna kinetycznie i~luźno związana (np. z~cząsteczkami wody lub niskoczasteczkowymi ligandami), a nie ściśle związana z~makrocząsteczkami---wolny Zn(II) jest dostępny dla makrocząsteczek. W~literaturze spotykane są synonimy: labilny Zn(II), wymienialny Zn(II), biodostępny Zn(II).}
	}

\newglossaryentry{rcsb}{type=\acronymtype,
name={RCSB PDB},
description={Research Collaboratory for Structural Bioinformatics Protein Data Bank},
first={Research Collaboratory for Structural Bioinformatics \gls{pdb} (RCSB \gls{pdb})}
}

\newglossaryentry{UniProt}{name={UniProt},
description={Universal Protein Resource (UniProt) to baza danych zawierająca
dane dotyczących sekwencji i~adnotacji białek.},
}


\newglossaryentry{pdb}{type=\acronymtype, name={PDB},
description={Protein Data Bank lub typ pliku Protein Data Bank},
first={Protein Data Bank (PDB)}
}

\newglossaryentry{ascii}{type=\acronymtype, name={ASCII},
description={Amerykański Standardowy Kod Wymiany Informacji
(ang. American Standard Code for Information Interchange)},
first={Amerykański Standardowy Kod Wymiany Informacji
 (ASCII, ang. American Standard Code for Information Interchange)}
}

\newglossaryentry{html}{type=\acronymtype, name={HTML},
description={hipertekstowy język znaczników (ang. HyperText Markup Language)},
first={hipertekstowy język znaczników (HTML, ang. HyperText Markup Language)}
}


\newglossaryentry{python}{name={Python},
	description={Python jest wysokopoziomowym, obiektowym językiem programowania
	często używanym w środowisku naukowym.}}




\newglossaryentry{sql}{type=\acronymtype,
name={SQL},
description={strukturalny język zapytań (ang. Structured Query Language)},
first={strukturalny język zapytań (SQL, ang. Structured Query Language)} 
}


\newglossaryentry{django}{name={Django},
	description={Django to wysokopoziomowa platforma programistyczna służąca do rozwoju aplikacji internetowych opartych na języku Python.}
	}

\newglossaryentry{bassembly}{name={złożenie biologiczne},
description={Złożenie biologiczne to układ makrocząsteczek
	w przestrzeni, co do którego uważa się lub udowodniono, że jest funkcjonalną formą cząsteczki.
	Złożenie biologiczne może być częścią jednostki asymetrycznej, jak również jednostka asymetryczna może zawierać jedną lub więcej kopii złożenia biologicznego.},
plural = {złożenia biologiczne}
}

\newglossaryentry{asunit}{name={jednostka asymetryczna},
description={Jednostka asymetryczna jest podstawową jednostką struktury krystalicznej. Po zastosowaniu
operacji symetrii na jednostce asymetrycznej można wygenerować komórkę krystalizacji, która powielona i~przełożona może dać obraz całego kryształu.}
}

\newglossaryentry{code}{name={kod źródłowy},
description={Kod źródłowy to zbiór instrukcji napisanych za pomocą języka programowania (np. za pomocą języka Python).}
}

\newglossaryentry{fold}{name={zwinięcie białkowe},
description={Definicja zwinięcia białkowego jest bardzo ogólna i~zależy od kontekstu.
Zwykle termin "zwinięcie białkowe" jest rozumiany jako wspólny kształt dla białek (lub domen białkowych).
Przykładem może być zwinięcie globinowe lub zwinięcie Rossmanna.}
}



\newglossaryentry{radg}{name={Rad50},
description={Komponent kompleksu MRN(X). Kluczowego kompleksu biorącego udział w naprawie DSB.}
}

\newglossaryentry{rad}{type=\acronymtype,
	name={Rad50},
	description={białko naprawy DNA Rad50, \glsseeformat[Słownik terminów:]{radg}{}},
	first={białko naprawy DNA (Rad50)\glsadd{radg}},
}


\newglossaryentry{mrng}{name={MRN(X)},
description={Kompleks składający się z białek Mre11, Rad50, oraz Nbs1 (lub Xrs2).
MRN(X) odgrywa kluczową rolę w naprawie DSB,
poprzedzając naprawę na drodze \glslink{nhej}{NHEJ} oraz \glslink{hr}{HR}.}
}

\newglossaryentry{mrn}{type=\acronymtype,
name={MRN(X)},
description={\glslink{mre}{Mre11}-\glslink{rad}{Rad50}-\glslink{nbs}{Nbs1}(\glslink{xrs}{Xrs2}) \glsseeformat[Słownik terminów:]{mrng}{}},
first={kompleks \glslink{mre}{Mre11}-\glslink{rad}{Rad50}-\glslink{nbs}{Nbs1}(\glslink{xrs}{Xrs2}) (MRN(X))\glsadd{mrng}},
}

\newglossaryentry{mr}{type=\acronymtype,
name={MR},
description={kompleks rdzeniowy \glslink{mre}{Mre11}--\glslink{rad}{Rad50} \glsseeformat[Słownik terminów:]{mrg}{}},
first={kompleks rdzeniowy \glslink{mre}{Mre11}--\glslink{rad}{Rad50} (MR)\glsadd{mrg}},
}

\newglossaryentry{mrg}{name={MR},
description={Kompleks rdzeniowy \glslink{mre}{Mre11}--\glslink{rad}{Rad50} odgrywa podobne role w bakeriach i~archeonach, pomimo braku \gls{nbs} lub \gls{xrs}.}
}


\newglossaryentry{mreg}{name={Mre11},
description={Komponent kompleksu \gls{mrn}. Odpowiedzialny za aktywność enzymatycznej jednoniciowej endonukleazy DNA i~specyficznej dla podwójnej nici DNA aktywności egzonukleazy 3'-5'.}
}

\newglossaryentry{mre}{type=\acronymtype,
name={Mre11},
description={białko naprawy pęknięć podwójnej nici DNA Mre11, \glsseeformat[Słownik terminów:]{mreg}{}},
first={białko naprawy pęknięć podwójnej nici DNA Mre11 (Mre11)\glsadd{mreg}},
}


\newglossaryentry{nbsg}{name={Nbs1},
description={Komponent kompleksu \gls{mrn}. Nbs1 moduluje naprawę uszkodzeń DNA poprzez rekrutację różnych kinaz.}
}

\newglossaryentry{nbs}{type=\acronymtype,
name={Nbs1},
description={nibryna, \glsseeformat[Słownik terminów:]{nbsg}{}},
first={nibryna (Nbs1)\glsadd{nbsg}},
}



\newglossaryentry{xrsg}{name={Xrs2},
description={Składnik kompleksu \gls{mrn} w \gls{sc}.
\Gls{xrs} i~\gls{nbs} mają niewielką identyczność sekwencji,
\gls{xrs} prawdopodobnie reguluje aktywność 5'-3' egzonukleazy kompleksu \gls{mrn}.}
}

\newglossaryentry{xrs}{type=\acronymtype,
name={Xrs2},
description={białko naprawy DNA Xrs2, \glsseeformat[Słownik terminów:]{xrsg}{}},
first={białko naprawy DNA Xrs2 (Xrs2)\glsadd{xrsg}}
}

\newglossaryentry{dsb}{type=\acronymtype,
name={DSB},
description={przerwanie podwójnej nici (ang. double-strand break)},
first={przerwanie podwójnej nici (DSB, ang. double-strand break)},
plural={DSBs},
descriptionplural={przerwania podwójnej nici (ang. double-strand breaks)},
firstplural={przerwania podwójnej nici (DSBs, ang. double-strand breaks)}
}

\newglossaryentry{hs}{type=\acronymtype,
name={\textit{H. sapiens}},
description={\textit{Homo sapiens}},
first={\textit{Homo sapiens} (\textit{H. sapiens})}
}

\newglossaryentry{pf}{type=\acronymtype,
name={\textit{P. furiosus}},
description={\textit{Pyrococcus furiosus}},
first={\textit{Pyrococcus furiosus} (\textit{P. furiosus})}
}

\newglossaryentry{sc}{type=\acronymtype,
name={\textit{S. cerevisiae}},
description={\textit{Saccharomyces cerevisiae}},
first={\textit{Saccharomyces cerevisiae} (\textit{S. cerevisiae})}
}

\newglossaryentry{ec}{type=\acronymtype,
name={\textit{E. coli}},
description={\textit{Escherichia coli}},
first={\textit{Escherichia coli} (\textit{E. coli})}
}

\newglossaryentry{cthermo}{type=\acronymtype,
name={\textit{C. thermophilum}},
description={\textit{Chaetomium thermophilum}},
first={\textit{Chaetomium thermophilum} (\textit{C. thermophilum})}
}


\newglossaryentry{if}{type=\acronymtype,
name={IF},
description={współczynnik oddziaływania (ang. impact factor)},
first={współczynnik oddziaływania (ang. impact factor)}
}


\newglossaryentry{cd}{type=\acronymtype,
name={CD},
description={dichroizm kołowy},
first={dichroizm kołowy (CD, ang. Circular Dichroism)}
}

\newglossaryentry{saxs}{type=\acronymtype,
name={SAXS},
description={rozpraszanie promieniowania rentgenowskiego pod małym kątem},
first={rozpraszanie promieniowania rentgenowskiego pod małym kątem (SAXS, ang. Small-angle X-ray scattering)}
}

\newglossaryentry{mals}{type=\acronymtype,
name={MALS},
description={wielokątowe rozpraszanie światła},
first={wielokątowe rozpraszanie światła (MALS, ang. multiangle light scattering)}
}

\newglossaryentry{xas}{type=\acronymtype,
name={XAS},
description={spektroskopia absorpcji promieniowania rentgenowskiego},
first={spektroskopia absorpcji promieniowania rentgenowskiego (XAS, ang. X-ray absorption spectroscopy)}
}

\newglossaryentry{hsabg}{name={HSAB},
	description={Zasada twardych i~miękkich kwasów  i~zasad opisuje, które kwasy i~zasady Lewisa preferują interakcje.
	Przykładowo miękkie jony metali (kwasy Lewisa) będą preferencyjnie oddziaływać z~miękkimi zasadami Lewisa, np., Hg(II) będzie preferencyjnie reagować z~donorami siarkowymi.}
	}

\newglossaryentry{hsab}{type=\acronymtype,
	name={HSAB},
	description={zasada twardych i~miękkich kwasów i~zasad (ang. Hard and Soft Acid and Bases), \glsseeformat[Słownik terminów:]{hsabg}{}},
	first={zasada twardych i~miękkich kwasów i~zasad (HSAB, ang. Hard and Soft Acid and Bases)\glsadd{hsabg}},
	}

\newglossaryentry{merr}{type=\acronymtype,
name={MerR},
description={białko regulacyjne operonu odporności na rtęć},
first={białko regulacyjne operonu odporności na rtęć (MerR)},
}

\newglossaryentry{atox}{type=\acronymtype,
name={ATOX1},
description={białko transportujące miedź ATOX1},
first={białko transportujące miedź ATOX1 (ATOX1)},
}

%!!
\newglossaryentry{uvvis}{type=\acronymtype,
name={UV--Vis},
description={zakres ultrafioletu i~światła widzialnego},
first={zakres ultrafioletu i~światła widzialnego (UV--Vis)}
}

\newglossaryentry{pac}{type=\acronymtype,
name={PAC},
description={zaburzona korelacja kątowa \textgamma-\textgamma\ (ang. perturbed \textgamma-\textgamma\ angular correlation)},
first={zaburzona korelacja kątowa \textgamma-\textgamma\ (PAC, ang. perturbed \textgamma-\textgamma\ angular correlation)}
}



\newglossaryentry{DTT}{type=\acronymtype,
name={DTT},
description={DL-ditiotreitol},
first={DL-ditiotreitol (DTT)}
}

\newglossaryentry{itc}{type=\acronymtype,
name={ITC},
description={izotermiczna kalorymetria miareczkowa (ang. isothermal titration calorimetry)},
first={izotermiczna kalorymetria miareczkowa (ITC, ang. isothermal titration calorimetry)}
}

\newglossaryentry{MS}{type=\acronymtype,
name={MS},
description={spektrometria mas},
first={spektrometria mas (MS)}
}

\newglossaryentry{dG}{type=\acronymtype,
name={\textDelta \textit{G}\textdegree},
description={zmiana energii swobodnej Gibbsa w stanie standardowym},
first={zmiana energii swobodnej Gibbsa w stanie standardowym (\textDelta \textit{G}\textdegree)}
}

\newglossaryentry{dH}{type=\acronymtype,
name={\textDelta \textit{H}\textdegree},
description={zmiana entalpii w stanie standardowym},
first={zmiana entalpii w stanie standardowym (\textDelta \textit{H}\textdegree)}
}

\newglossaryentry{dS}{type=\acronymtype,
name={\textDelta \textit{S}\textdegree},
description={zmiana entropii w stanie standardowym},
first={zmiana entropii w stanie standardowym (\textDelta \textit{S}\textdegree)}
}

\newglossaryentry{T}{type=\acronymtype,
name={\textit{T}},
description={temperatura bezwzględna},
first={temperatura bezwzględna (\textit{T})}
}

\newglossaryentry{K12}{type=\acronymtype,
name={\textit{K}\textsubscript{12}},
description={kumulatywna stała tworzenia się  kompleksu \glslink{Me}{M}\glslink{L}{L}\textsubscript{2} },
first={kumulatywna stała tworzenia się  kompleksu \glslink{Me}{M}\glslink{L}{L}\textsubscript{2}  (\textit{K}\textsubscript{12})}
}

\newglossaryentry{Ka}{type=\acronymtype,
name={\textit{K}\textsubscript{a}},
description={stała asocjacji},
first={stała asocjacji (\textit{K}\textsubscript{a})}
}



\newglossaryentry{Kd}{type=\acronymtype,
name={\textit{K}\textsubscript{d}},
description={stała dysocjacji},
first={stała dysocjacji (\textit{K}\textsubscript{d})}
}


\newglossaryentry{Kex}{type=\acronymtype,
name={\textit{K}\textsubscript{ex}},
description={stała wymiany},
first={stała wymiany (\textit{K}\textsubscript{ex})}
}

\newglossaryentry{koff}{type=\acronymtype,
name={\textit{k}\textsubscript{off}},
description={kinetyczna stała dysocjacji},
first={kinetyczna stała dysocjacji (\textit{k}\textsubscript{off})}
}

\newglossaryentry{kon}{type=\acronymtype,
name={\textit{k}\textsubscript{on}},
description={kinetyczna stała asocjacji},
first={kinetyczna stała asocjacji (\textit{k}\textsubscript{on})}
}



\newglossaryentry{lmct}{type=\acronymtype,
name={LMCT},
description={przeniesienie ładunku od liganda do jonu metalu (ang. ligand-metal charge transfer)},
first={ przeniesienie ładunku od liganda do jonu metalu (LMCT, ang. ligand-metal charge transfer)}
}

\newglossaryentry{ct}{type=\acronymtype,
name={CT},
description={przeniesienie ładunku (ang. charge transfer)},
first={przeniesienie ładunku (CT, ang. charge transfer)}
}


\newglossaryentry{Me}{type=\acronymtype,
name={M},
description={metal lub jon metalu},
first={metal lub jon metalu (M)}
}

\newglossaryentry{Lg}{name={ligand},
	description={W chemii ligand to jon lub cząsteczka, będąca zasadą Lewisa, oddziałująca
	z~kwasem Lewisa poprzez wiązanie koordynacyjne. (patrz: \gls{hsab}).
	W biochemii ligand rozumiany jest jako cząsteczka, która wiąże się z~inną
	(makro)cząsteczką.}
	}

\newglossaryentry{L}{type=\acronymtype, name={L},
	description={ligand, \glsseeformat[Słownik terminów:]{Lg}{}},
	first={ligand (L) \glsadd{Lg}},
	descriptionplural ={ligandy}
	}

\newglossaryentry{C}{type=\acronymtype, name={C},
description={ligand współzawodniczacy (C)},
first={ligand współzawodniczacy (C)},
descriptionplural ={ligandy współzawodniczace}
}

\newglossaryentry{R}{type=\acronymtype,
name={R},
description={molowa stała~gazowa},
first={molowa stała~gazowa (R)}
}

\newglossaryentry{agnp}{type=\acronymtype, name={AgNP},
	description={nanocząstka srebra},
	first={nanocząstka srebra (AgNP, ang. silver nanoparticle)},
	descriptionplural={nanocząstki srebra},
	firstplural = {nanocząstki srebra (AgNPs, ang. silver nanoparticles)}
	}

\newglossaryentry{hmw}{type=\acronymtype, name={HMW},
description={high molecular weight},
first={high molecular weight (HMW)},
}

\newglossaryentry{lmw}{type=\acronymtype, name={LMW},
description={niska masa cząsteczkowa},
first={niska masa cząsteczkowa (LMW, ang. low molecular weight)},
}


\newglossaryentry{ptmg}{name={PTM},
	description={modyfikacja potranslacyjna to kowalencyjna (zwykle enzymatyczna)
	modyfikacja białka, która następuje po biosyntezie białka.}
	}

\newglossaryentry{ptm}{type=\acronymtype, name={PTM},
	description={modyfikacja potranslacyjna, \glsseeformat[Słownik terminów:]{ptmg}{}},
	first={modyfikacja potranslacyjna (PTM, ang. post-translational modification) \glsadd{ptmg}},
	plural={PTMs},
	descriptionplural={modyfikacje potranslacyjne},
	firstplural={modyfikacje potranslacyjne (PTMs, ang. post-translational modifications)}
	}

\newglossaryentry{coenzyme}{name={koenzym},
description={Koenzym jest niebiałkowym organicznym kofaktorem.}
} %https://www.jove.com/science-education/10975/cofactors-and-coenzymes

\newglossaryentry{cofactor}{name={kofaktor},
description={Kofaktor to niebiałkowy związek chemiczny, atom lub jon, wymagany aby białko mogło spełniać swoje funkcje.}
} 

\newglossaryentry{pgroup}{name={grupa prostetyczna},
description={Grupa prostetyczna to związek niebiałkowy niezbędny do pełnienia przez białko jego funkcji, jednak w przeciwieństwie do kofaktora grupa prostetyczna jest kowalencyjnie związana z~białkiem.}
} 


\newglossaryentry{mprotein}{name={metalobiałko},
description={Metalobiałko to białko, które zawiera co najmniej jeden jon metalu jako kofaktor lub jako część grupy prostetycznej.}
} 


\newglossaryentry{atp}{type=\acronymtype, name={ATP},
	description={adenozyno-5$\prime$-trifosforan},
	first={adenozyno-5$\prime$-trifosforan (ATP)},
	}

\newglossaryentry{sod}{type=\acronymtype, name={SOD},
description={dysmutaza ponadtlenkowa},
descriptionplural={dysmutaza ponadtlenkowa},
first={dysmutaza ponadtlenkowa (SOD, ang. superoxide dismutase)},
firstplural={dysmutazy ponadtlenkowe (SODs, ang. superoxide dismutase)},
}

\newglossaryentry{dna}{type=\acronymtype, name={DNA},
	description={kwas deoksyrybonukleinowy},
	first={kwas deoksyrybonukleinowy (DNA)},
	}

\newglossaryentry{ssdna}{type=\acronymtype, name={ssDNA},
description={jednoniciowe \glslink{dna}{DNA}},
first={jednoniciowe \glslink{dna}{DNA} (ssDNA, ang. single-stranded \glslink{dna}{DNA})},
}

\newglossaryentry{dsdna}{type=\acronymtype, name={dsDNA},
description={dwuniciowe \glslink{dna}{DNA}},
first={dwuniciowe \glslink{dna}{DNA} (dsDNA, ang. double-stranded \glslink{dna}{DNA})},
}

\newglossaryentry{ku}{type=\acronymtype, name={Ku},
description={białko łączące niehomologiczne końce Ku},
first={białko łączące niehomologiczne końce Ku (Ku)},
}



\newglossaryentry{ctip}{type=\acronymtype, name={CtiP},
description={białko oddziałujące z~CtBP},
first={białko oddziałujące z~ CtBP (CtiP, ang. CtBP-interacting protein)},
}

\newglossaryentry{sae}{type=\acronymtype, name={Sae2},
description={endonukleaza DNA  Sae2},
first={endonukleaza DNA  Sae2 (Sae2)},
}



\newglossaryentry{ca2}{type=\acronymtype, name={CAII},
description={anhydraza węglowa II},
first={anhydraza węglowa II (CAII, ang. carbonic anhydrase II)},
}

\newglossaryentry{zf}{type=\acronymtype, name={ZF},
description={palec cynkowy},
first={palec cynkowy  (ZF, ang. zinc finger)},
descriptionplural = {palce cynkowe},
firstplural = {palce cynkowe  (ZFs, ang. zinc fingers)}
}


\newglossaryentry{cn}{type=\acronymtype, name={LK},
description={liczba koordynacyjna},
first={liczba koordynacyjna  (LK)},
firstplural={liczby koordynacyjna  (LK)},
plural={LK},
}

\newglossaryentry{OSETA}{type=\acronymtype,
name={OSETA},
description={pojedyncze oczyszczanie powinowactwa ze znacznikiem jedno epitopowym (ang. one single-epitope tag affinity purification)},
first={pojedyncze oczyszczanie powinowactwa ze znacznikiem jedno epitopowym (OSETA, ang. one single-epitope tag affinity purification)}
}

\newglossaryentry{TAP}{type=\acronymtype,
name={TAP},
description={tandemowa chromatografia powinowactwa (ang. tandem affinity purification)},
first={tandemowa chromatografia powinowactwa  (TAP, ang. tandem affinity purification)}
}

\newglossaryentry{SEC}{type=\acronymtype,
name={SEC},
description={chromatografia wykluczania (ang. size exclusion chromatography)},
first={chromatografia wykluczania (SEC, ang. size exclusion chromatography)}
}

\newglossaryentry{FRET}{type=\acronymtype,
name={FRET},
description={transfer energii rezonansu Förstera (ang. Förster resonance energy transfer)},
first={transfer energii rezonansu Förstera (FRET, ang. Förster resonance energy transfer)}
}

\newglossaryentry{BLI}{type=\acronymtype,
name={BLI},
description={interferometria warstwy biologicznej (ang. Bio-Layer Interferometry)},
first={interferometria warstwy biologicznej (BLI, ang. Bio-Layer Interferometry)}
}


\newglossaryentry{SPR}{type=\acronymtype,
name={SPR},
description={rezonans plazmonów powierzchniowych (ang. surface plasmon resonance)},
first={rezonans plazmonów powierzchniowych (SPR, ang. surface plasmon resonance)}
}

\newglossaryentry{lck}{type=\acronymtype,
name={Lck},
description={specyficzna dla limfocytów białkowa kinaza tyrozynowa (ang. lymphocyte-specific protein tyrosine kinase)},
first={specyficzna dla limfocytów białkowa kinaza tyrozynowa (Lck, ang. lymphocyte-specific protein tyrosine kinase)}
}

\newglossaryentry{cd4}{type=\acronymtype,
name={CD4},
description={glikoproteina CD4 (ang. cluster of differentiation 4)},
first={glikoproteina CD4 (CD4, ang. cluster of differentiation 4)}
}


\newglossaryentry{cd8}{type=\acronymtype,
name={CD8},
description={glikoproteina CD8 (ang. cluster of differentiation 8)},
first={glikoproteina CD8 (CD4, ang. cluster of differentiation 8)}
}


\newglossaryentry{nmr}{type=\acronymtype,
name={NMR},
description={magnetyczny rezonans jądrowy (ang. nuclear magnetic resonance)},
first={magnetyczny rezonans jądrowy, (NMR, ang. nuclear magnetic resonance)}
}

\newglossaryentry{cryoem}{type=\acronymtype,
name={cryoEM},
description={kriogeniczna mikroskopia elektronowa (ang. cryogenic electron microscopy)},
first={kriogeniczna mikroskopia elektronowa (cryoEM, ang. cryogenic electron microscopy)}
}


\newglossaryentry{exafs}{type=\acronymtype,
name={EXAFS},
description={rozszerzona subtelna struktura  absorpcji promieniowania rentgenowskiego (ang. extended X-ray absorption fine structure)},
first={rozszerzona subtelna struktura absorpcji promieniowania rentgenowskiego (EXAFS, ang. extended X-ray absorption fine structure)}
}

\newglossaryentry{xanes}{type=\acronymtype,
name={XANES},
description={struktura absorpcji promieniowania rentgenowskiego w pobliżu krawędzi (ang. X-ray absorption near-edge structure)},
first={struktura absorpcji promieniowania rentgenowskiego w pobliżu krawędzi (XANES, ang. X-ray absorption near-edge structure)}
}

\newglossaryentry{fsd1}{type=\acronymtype,
name={FSD-1},
description={peptyd Full Sequence Design 1},
first={peptyd Full Sequence Design 1 (FSD-1)}
}

\newglossaryentry{zif268}{type=\acronymtype,
name={Zif268},
description={czynnik transkrypcyjny zawierający palec cynkowy 268 (ang. zinc finger-containing transcription factor 268)},
first={czynnik transkrypcyjny zawierający palec cynkowy 268 (Zif268, ang. zinc finger-containing transcription factor 268)}
}

\newglossaryentry{mukb}{type=\acronymtype,
name={MukB},
description={białko podziału chromosomu MukB},
first={białko podziału chromosomu MukB (MukB)}
}

\newglossaryentry{zap1}{type=\acronymtype,
name={ZAP1},
description={regulator transkrypcyjny reagujący na cynk ZAP1},
first={regulator transkrypcyjny reagujący na cynk ZAP1 (ZAP1)}
}


\newglossaryentry{zrt1}{type=\acronymtype,
name={ZRT1},
description={transporter regulowany przez cynk 1 (ang. zinc-regulated transporter 1)},
first={transporter regulowany przez cynk (ZRT1, ang. zinc-regulated transporter 1)}
}

\newglossaryentry{zrt2}{type=\acronymtype,
name={ZRT2},
description={transporter regulowany przez cynk 2 (ang. zinc-regulated transporter 2)},
first={transporter regulowany przez cynk (ZRT2, ang. zinc-regulated transporter 2)}
}


\newglossaryentry{zrt3}{type=\acronymtype,
name={ZRT3},
description={transporter regulowany przez cynk 3 (ang. zinc-regulated transporter 3)},
first={transporter regulowany przez cynk (ZRT3, ang. zinc-regulated transporter 3)}
}


\newglossaryentry{zip}{type=\acronymtype,
name={ZIP},
description={białka podobne do Zrt i~Irt (ang. Zrt-, Irt-like proteins)},
first={białka podobne do Zrt i~Irt (ZIP, ang. Zrt-, Irt-like proteins)}
}


\newglossaryentry{nhejg}{name={NHEJ},
	description={Niehomologiczne łączenie końców, jest ścieżką naprawy przerwania podwójnej nici DNA poprzez ligację końców bez potrzeby istnienia homologicznego wzorca.}}

\newglossaryentry{nhej}{type=\acronymtype, name={NHEJ},
	description={niehomologiczne łączenie końców (ang. non-homologous end joining), \glsseeformat[Słownik terminów:]{nhejg}{}},
	first={niehomologiczne łączenie końców (NHEJ, ang. non-homologous end joining) \glsadd{nhejg}},
	plural={NHEJs},
	descriptionplural={niehomologiczne łączenie końców (ang. non-homologous end joining)},
	firstplural={niehomologiczne łączenie końców (NHEJs, ang. non-homologous end joining) (\glsentryplural{nhej})}}



\newglossaryentry{hrg}{name={HR},
description={Homologiczne łączenie końców,
 jest ścieżką naprawy przerwania podwójnej nici DNA, z~wykorzystaniem
 siostrzanej chromatydy lub homologicznego chromosomu jako wzorca naprawy.}}

\newglossaryentry{hr}{type=\acronymtype, name={HR},
	description={rekombinacja homologiczna (ang. homologous recombination) \glsseeformat[Słownik terminów:]{hrg}{}},
	first={rekombinacja homologiczna (HR, ang. homologous recombination)\glsadd{hrg}},
	plural={HRs},
	descriptionplural={rekombinacja homologiczna (HR, ang. homologous recombination)},
	firstplural={rekombinacja homologiczna (HRs, ang. homologous recombination) }}


\newglossaryentry{zn}{type=\acronymtype,
name={Zn(II)},
description={cynk(II))},
first={cynk(II) (Zn(II))}
}

\newglossaryentry{hg}{type=\acronymtype,
name={Hg(II)},
description={rtęć(II))},
first={rtęć(II) (Hg(II))}
}


\newglossaryentry{ag}{type=\acronymtype,
name={Ag(I)},
description={srebro(I)},
first={srebro(I) (Ag(I))}
}


%%%% shortcuts for glossaries
\newcommand{\gshort}[1]{\glslink{#1}{#1}}
\newcommand{\glink}[2]{\glslink{#1}{#2}\glsunset{#1}}
% affinity

% ****************************************************************************************************
% 4. Setup floats: tables, (sub)figures, and captions
% ****************************************************************************************************
\usepackage{tabularx} % better tables
\setlength{\extrarowheight}{3pt} % increase table row height
\newcommand{\tableheadline}[1]{\multicolumn{1}{l}{\spacedlowsmallcaps{#1}}}
\usepackage{subfig}
% ****************************************************************************************************


% ****************************************************************************************************
% 5. Setup code listings
% ****************************************************************************************************
\usepackage{listings}
%\lstset{emph={trueIndex,root},emphstyle=\color{BlueViolet}}%\underbar} % for special keywords
\lstset{language=[LaTeX]Tex,%C++,
	keywordstyle=\color{RoyalBlue},%\bfseries,
	basicstyle=\small\ttfamily,
	%identifierstyle=\color{NavyBlue},
	commentstyle=\color{Green}\ttfamily,
	stringstyle=\rmfamily,
	numbers=none,%left,%
	numberstyle=\scriptsize,%\tiny
	stepnumber=5,
	numbersep=8pt,
	showstringspaces=false,
	breaklines=true,
	frameround=ftff,
	frame=single,
	belowcaptionskip=.75\baselineskip
	%frame=L
}
% ****************************************************************************************************


% ****************************************************************************************************
% 6. Last calls before the bar closes
% ****************************************************************************************************
% ********************************************************************
% Her Majesty herself
% ********************************************************************
\usepackage{classicthesis}


% ********************************************************************
% Fine-tune hyperreferences (hyperref should be called last)
% ********************************************************************
\hypersetup{%
	%draft, % hyperref's draft mode, for printing see below
	colorlinks=true, linktocpage=true, pdfstartpage=3, pdfstartview=FitV,%
	% uncomment the following line if you want to have black links (e.g., for printing)
	%colorlinks=false, linktocpage=false, pdfstartpage=3, pdfstartview=FitV, pdfborder={0 0 0},%
	breaklinks=true, pdfpagemode=UseNone, pageanchor=true, pdfpagemode=UseNone,%
	plainpages=false, bookmarksnumbered, bookmarksopen=true, bookmarksopenlevel=1,%
	hypertexnames=true, pdfhighlight=/O,%nesting=true,%frenchlinks,%
	urlcolor=CTurl, linkcolor=CTlink, citecolor=CTcitation,%
	%urlcolor=Black, linkcolor=Black, citecolor=Black,%
	pdftitle={\myTitle},%
	pdfauthor={\textcopyright\ \myName, \myUni, \myFaculty},%
	pdfsubject={},%
	pdfkeywords={},%
	pdfcreator={},%
	pdfproducer={LaTeX with classicthesis style}%
}


% ********************************************************************
% Setup autoreferences (hyperref and babel)
% ********************************************************************
% There are some issues regarding autorefnames
% http://www.ureader.de/msg/136221647.aspx
% http://www.tex.ac.uk/cgi-bin/texfaq2html?label=latexwords
% you have to redefine the macros for the
% language you use, e.g., american, ngerman
% (as chosen when loading babel/AtBeginDocument)
% ********************************************************************
\makeatletter
\@ifpackageloaded{babel}%
{%
	\addto\extrasamerican{%
		\renewcommand*{\figureautorefname}{Figure}%
		\renewcommand*{\tableautorefname}{Table}%
		\renewcommand*{\partautorefname}{Part}%
		\renewcommand*{\chapterautorefname}{Chapter}%
		\renewcommand*{\sectionautorefname}{Section}%
		\renewcommand*{\subsectionautorefname}{Section}%
		\renewcommand*{\subsubsectionautorefname}{Section}%
	}%
	\addto\extraspolish{%
		\renewcommand*{\figureautorefname}{Figura}%
		\renewcommand*{\tableautorefname}{Tabela}%
		\renewcommand*{\partautorefname}{Część}%
		\renewcommand*{\chapterautorefname}{Rozdział}%
		\renewcommand*{\sectionautorefname}{Sekcja}%
		\renewcommand*{\subsectionautorefname}{Sekcja}%
		\renewcommand*{\subsubsectionautorefname}{Sekcja}%
		\renewcommand*{\equationautorefname}{Równanie}%
	}%
	\addto\extrasngerman{%
		\renewcommand*{\paragraphautorefname}{Absatz}%
		\renewcommand*{\subparagraphautorefname}{Unterabsatz}%
		\renewcommand*{\footnoteautorefname}{Fu\"snote}%
		\renewcommand*{\FancyVerbLineautorefname}{Zeile}%
		\renewcommand*{\theoremautorefname}{Theorem}%
		\renewcommand*{\appendixautorefname}{Anhang}%
		\renewcommand*{\equationautorefname}{Gleichung}%
		\renewcommand*{\itemautorefname}{Punkt}%
	}%
	% Fix to getting autorefs for subfigures right (thanks to Belinda Vogt for changing the definition)
	\providecommand{\subfigureautorefname}{\figureautorefname}%
}{\relax}
\makeatother

% ****************************************************************************************************
% 7. Further adjustments (experimental)
% ****************************************************************************************************
\addto\captionspolish{%
  \renewcommand{\figurename}{Figura}%
}


% ********************************************************************
% Changing the text area
% ********************************************************************
%\areaset[current]{312pt}{761pt} % 686 (factor 2.2) + 33 head + 42 head \the\footskip
%\setlength{\marginparwidth}{7em}%
%\setlength{\marginparsep}{2em}%

% ********************************************************************
% Using different fonts
% This is for pdflatex; xelatex and lualatex have their own way
% ********************************************************************
%\usepackage[oldstylenums]{kpfonts} % oldstyle notextcomp
%\usepackage[osf]{libertine}
%\usepackage[light,condensed,math]{iwona}
%\renewcommand{\sfdefault}{iwona}
%\usepackage{lmodern} % <-- no osf support :-(
%\usepackage{cfr-lm} %
%\usepackage[urw-garamond]{mathdesign} <-- no osf support :-(
%\usepackage[default,osfigures]{opensans} % scale=0.95
%\usepackage[sfdefault]{FiraSans}
%\usepackage[opticals,mathlf]{MinionPro} % onlytext
% ********************************************************************
% \usepackage[largesc,osf]{newpxtext}
\linespread{1.05} % a bit more for Palatino
% Used to fix these:
% https://bitbucket.org/amiede/classicthesis/issues/139/italics-in-pallatino-capitals-chapter
% https://bitbucket.org/amiede/classicthesis/issues/45/problema-testatine-su-classicthesis-style
% ********************************************************************
%\linespread{1.05} % a bit more for Palatino
% ****************************************************************************************************
