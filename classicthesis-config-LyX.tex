% ****************************************************************************************************
% classicthesis-config-LyX.tex
% Use it in the preamble of your ClassicThesis.lyx
% ****************************************************************************************************
% If you like the classicthesis, we would appreciate a postcard.
% André's address can be found in the file ClassicThesis.pdf. A collection
% of the postcards received so far is available online at
% http://postcards.miede.de
% ****************************************************************************************************


% ****************************************************************************************************
% 1. Configure classicthesis for your needs here. At some point you will probably want to set 
% "drafting" to false in order to deactivate the time-stamp on the bottom of the pages
% See ClassicThesis.pdf for more information.
% ****************************************************************************************************
\PassOptionsToPackage{
	drafting=true,      % print version information on the bottom of the pages
	tocaligned=false,   % the left column of the toc will be aligned (no indentation)
	dottedtoc=false,    % page numbers in ToC flushed right
	parts=true,         % use part division
	eulerchapternumbers=true, % use AMS Euler for chapter font (otherwise Palatino)
	linedheaders=false,       % chaper headers will have line above and beneath
	floatperchapter=true,     % numbering per chapter for all floats (i.e., Figure 1.1)
	eulermath=false,    % use awesome Euler fonts for mathematical formulae (only with pdfLaTeX)
	beramono=true,      % toggle a nice monospaced font (w/ bold)
	palatino=false,      % toggle the standard roman font, see end of this file for more suggestions
	style=classicthesis % classicthesis, arsclassica
}{classicthesis}


% ****************************************************************************************************
% 2. Personal data and user ad-hoc commands
% ****************************************************************************************************
\newcommand{\myTitle}{TITLE\xspace}
\newcommand{\mySubtitle}{Doctoral dissertation\xspace}
\newcommand{\myDegree}{Master (MSc)\xspace}
\newcommand{\myName}{Józef Tran\xspace}
\newcommand{\myProf}{Artur Krężel\xspace}
% \newcommand{\myOtherProf}{Put name here\xspace}
% \newcommand{\mySupervisor}{Put name here\xspace}
\newcommand{\myFaculty}{Wydział Biotechnologii\xspace}
\newcommand{\myDepartment}{Department of Chemical Biology\xspace}
\newcommand{\myUni}{Uniwersytet Wrocławski\xspace}
\newcommand{\myLocation}{Wrocław\xspace}
\newcommand{\myTime}{February 2023\xspace}
\newcommand{\myVersion}{v1.0}

% There is a problem using ę with Mendeley, ę was expressed as \c{e}
\DeclareUnicodeCharacter{0229}{\k{e}}

% ****************************************************************************************************


% ****************************************************************************************************
% 3. Loading some handy packages
% ****************************************************************************************************
\usepackage{csquotes} % biblatex depends on these

\usepackage{scrhack} % fix warnings when using KOMA with listings package
\usepackage{xspace} % to get the spacing after macros right
\usepackage{mparhack} % get marginpar right
\PassOptionsToPackage{printonlyused,smaller}{acronym}
\usepackage{acronym} % nice macros for handling all acronyms in the thesis
%\renewcommand{\bflabel}[1]{{#1}\hfill} % fix the list of acronyms --> no longer working
%\renewcommand*{\acsfont}[1]{\textsc{#1}} 
%\renewcommand*{\aclabelfont}[1]{\acsfont{#1}}
\def\bflabel#1{{\acsfont{#1}\hfill}}
\def\aclabelfont#1{\acsfont{#1}}

% Package to use Python code in the thesis 
\usepackage[gobble=auto]{pythontex}



\usepackage{newpxtext,newpxmath}
\useosf

% ****************************************************************************************************

% ****************************************************************************************************
% Glossary
% ****************************************************************************************************
% \usepackage[acronym]{glossaries}



\usepackage[automake,nolist,nopostdot,sort=use,acronym,]{glossaries}




\newglossarystyle{no_indentation_long}{% 
	\setglossarystyle{long}
	\renewenvironment{theglossary}%
	{\begin{longtable}[l]{@{}lp{\glsdescwidth}}}%
			{\end{longtable}}
}



%acronyms
%akronimy komentarze przykłady, to edytować - pozwala na robienie odnośników

\makeglossaries

%% modsuper glossary display style

% \newacronym{API}{API}{Application Programming Interface}
% \newglossaryentry{api}{type=\acronymtype, name={API}, description={Application Programming Interface}, first={Application Programming Interface (API)}}


\newglossaryentry{apig}{name={API},
	description={An Application Programming Interface (API) is a set
			of rules and protocols that a allows two or more software components
			to communicate with each other.}}

\newglossaryentry{api}{type=\acronymtype, name={API},
	description={Application
			Programming Interface, \glsseeformat[Glossary:]{apig}{}},
	first={Application Programming Interface (API)\glsadd{apig}}}




\newglossaryentry{ppi}{type=\acronymtype, name={PPI},
	description={Protein-Protein Interaction},
	first={Protein-Protein Interaction (PPI)},
	plural={PPIs},
	descriptionplural={Protein-Protein Interactions},
	firstplural={Protein-Protein Interactions (\glsentryplural{ppi})}}



\newglossaryentry{proteome}{name={proteome},
	description={A proteome is the complete set of poroteins
			that can or are expressed by an organism.},}

\newglossaryentry{htsg}{name={HTS},
	description={High-throughput screening is a application
			of robotics followed by data processing to test large
			number of samples.}}

\newglossaryentry{hts}{type=\acronymtype, name={HTS},
	description={High-Throughput Screening, \glsseeformat[Glossary:]{htsg}{}},
	first={High-Throughput Screening (HTS)\glsadd{htsg}}}

\newglossaryentry{interface}{name={interface},
	description={A shared surface between two or more components.
			Interface in \gls{ppi} is made of residues of a protein that
			contact with residues from another interacting partner. Sometimes
			interfaces may include additional interaction partners, e.g. protein cofactors
			and ligands.}}

\newglossaryentry{zppig}{name={ZPPI},
	description={Zn(II)-involved \glsdesc{ppi} is a protein-protein interaction
			in which a vital role is played by Zn(II)},}

\newglossaryentry{zppi}{type=\acronymtype, name={ZPPI},
	description={Zn(II)-involved \glsdesc{ppi}, \glsseeformat[Glossary:]{zppig}{}},
	first={Zn(II)-involved \glsdesc{ppi} (ZPPI)\glsadd{zppig}},
	plural={ZPPIs},
	descriptionplural={Zn(II)-involved \glsdescplural{ppi}},
  firstplural={Zn(II)-involved \glsdescplural{ppi} (\glsentryplural{zppi})}}

\newglossaryentry{cry}{type=\acronymtype, name={CRY1},
	description={Cryptochrome 1},
	first={Cryptochrome 1 (CRY1)},
}

\newglossaryentry{per}{type=\acronymtype, name={PER2},
	description={Period 2},
	first={Period 2 (PER2)}
}


\newglossaryentry{freezn}{name={free Zn(II)},
	description={Available, loosely bound Zn(II) ion.},}

\newglossaryentry{rcsb}{type=\acronymtype, name={RCSB PDB},
description={Research Collaboratory for Structural Bioinformatics Protein Data Bank},
first={Research Collaboratory for Structural Bioinformatics Protein Data Bank (RCSB PDB)}
}

\newglossaryentry{python}{name={Python},
	description={Python is a high-level, object-oriented programming
	language, often used in scientific community.},}

\newglossaryentry{bassembly}{name={biological assembly},
description={The biological assembly is the arrangement of macromolecules
in space that is believed or proved to be the functional form of the molecule.
The biological assembly may be a part of the \gls{asunit}, as well as \gls{asunit} may contain one
or more copies of the biological assembly.},
plural = {biological assemblies}}

\newglossaryentry{asunit}{name={assymetric unit},
description={The asymmetric unit is the basic unit ot a crystal structure. After application
symmetry operations to the asymmetric unit it is possible to generate the unit cell, which duplicated and translated can produce the whole crystal.},}

\newglossaryentry{code}{name={source code},
description={Source code is the set of instuctions written using programming language (e.g. \gls{python}.)},}

\newglossaryentry{fold}{name={protein fold},
description={Definition of a protein fold is arbitrary and depending on a context.
Generally protein fold is usually understood as a comon shape for proteins.}}


 
% affinity

% ****************************************************************************************************
% 4. Setup floats: tables, (sub)figures, and captions
% ****************************************************************************************************
\usepackage{tabularx} % better tables
\setlength{\extrarowheight}{3pt} % increase table row height
\newcommand{\tableheadline}[1]{\multicolumn{1}{l}{\spacedlowsmallcaps{#1}}}
\usepackage{subfig}
% ****************************************************************************************************


% ****************************************************************************************************
% 5. Setup code listings
% ****************************************************************************************************
\usepackage{listings}
%\lstset{emph={trueIndex,root},emphstyle=\color{BlueViolet}}%\underbar} % for special keywords
\lstset{language=[LaTeX]Tex,%C++,
	keywordstyle=\color{RoyalBlue},%\bfseries,
	basicstyle=\small\ttfamily,
	%identifierstyle=\color{NavyBlue},
	commentstyle=\color{Green}\ttfamily,
	stringstyle=\rmfamily,
	numbers=none,%left,%
	numberstyle=\scriptsize,%\tiny
	stepnumber=5,
	numbersep=8pt,
	showstringspaces=false,
	breaklines=true,
	frameround=ftff,
	frame=single,
	belowcaptionskip=.75\baselineskip
	%frame=L
}
% ****************************************************************************************************


% ****************************************************************************************************
% 6. Last calls before the bar closes
% ****************************************************************************************************
% ********************************************************************
% Her Majesty herself
% ********************************************************************
\usepackage{classicthesis}


% ********************************************************************
% Fine-tune hyperreferences (hyperref should be called last)
% ********************************************************************
\hypersetup{%
	%draft, % hyperref's draft mode, for printing see below
	colorlinks=true, linktocpage=true, pdfstartpage=3, pdfstartview=FitV,%
	% uncomment the following line if you want to have black links (e.g., for printing)
	%colorlinks=false, linktocpage=false, pdfstartpage=3, pdfstartview=FitV, pdfborder={0 0 0},%
	breaklinks=true, pdfpagemode=UseNone, pageanchor=true, pdfpagemode=UseNone,%
	plainpages=false, bookmarksnumbered, bookmarksopen=true, bookmarksopenlevel=1,%
	hypertexnames=true, pdfhighlight=/O,%nesting=true,%frenchlinks,%
	urlcolor=CTurl, linkcolor=CTlink, citecolor=CTcitation,%
	%urlcolor=Black, linkcolor=Black, citecolor=Black,%
	pdftitle={\myTitle},%
	pdfauthor={\textcopyright\ \myName, \myUni, \myFaculty},%
	pdfsubject={},%
	pdfkeywords={},%
	pdfcreator={},%
	pdfproducer={LaTeX with classicthesis style}%
}


% ********************************************************************
% Setup autoreferences (hyperref and babel)
% ********************************************************************
% There are some issues regarding autorefnames
% http://www.ureader.de/msg/136221647.aspx
% http://www.tex.ac.uk/cgi-bin/texfaq2html?label=latexwords
% you have to redefine the macros for the
% language you use, e.g., american, ngerman
% (as chosen when loading babel/AtBeginDocument)
% ********************************************************************
\makeatletter
\@ifpackageloaded{babel}%
{%
	\addto\extrasamerican{%
		\renewcommand*{\figureautorefname}{Figure}%
		\renewcommand*{\tableautorefname}{Table}%
		\renewcommand*{\partautorefname}{Part}%
		\renewcommand*{\chapterautorefname}{Chapter}%
		\renewcommand*{\sectionautorefname}{Section}%
		\renewcommand*{\subsectionautorefname}{Section}%
		\renewcommand*{\subsubsectionautorefname}{Section}%
	}%
	\addto\extrasngerman{%
		\renewcommand*{\paragraphautorefname}{Absatz}%
		\renewcommand*{\subparagraphautorefname}{Unterabsatz}%
		\renewcommand*{\footnoteautorefname}{Fu\"snote}%
		\renewcommand*{\FancyVerbLineautorefname}{Zeile}%
		\renewcommand*{\theoremautorefname}{Theorem}%
		\renewcommand*{\appendixautorefname}{Anhang}%
		\renewcommand*{\equationautorefname}{Gleichung}%
		\renewcommand*{\itemautorefname}{Punkt}%
	}%
	% Fix to getting autorefs for subfigures right (thanks to Belinda Vogt for changing the definition)
	\providecommand{\subfigureautorefname}{\figureautorefname}%
}{\relax}
\makeatother

% ****************************************************************************************************
% 7. Further adjustments (experimental)
% ****************************************************************************************************
% ********************************************************************
% Changing the text area
% ********************************************************************
%\areaset[current]{312pt}{761pt} % 686 (factor 2.2) + 33 head + 42 head \the\footskip
%\setlength{\marginparwidth}{7em}%
%\setlength{\marginparsep}{2em}%

% ********************************************************************
% Using different fonts
% This is for pdflatex; xelatex and lualatex have their own way
% ********************************************************************
%\usepackage[oldstylenums]{kpfonts} % oldstyle notextcomp
%\usepackage[osf]{libertine}
%\usepackage[light,condensed,math]{iwona}
%\renewcommand{\sfdefault}{iwona}
%\usepackage{lmodern} % <-- no osf support :-(
%\usepackage{cfr-lm} %
%\usepackage[urw-garamond]{mathdesign} <-- no osf support :-(
%\usepackage[default,osfigures]{opensans} % scale=0.95
%\usepackage[sfdefault]{FiraSans}
%\usepackage[opticals,mathlf]{MinionPro} % onlytext
% ********************************************************************
% \usepackage[largesc,osf]{newpxtext}
\linespread{1.05} % a bit more for Palatino
% Used to fix these:
% https://bitbucket.org/amiede/classicthesis/issues/139/italics-in-pallatino-capitals-chapter
% https://bitbucket.org/amiede/classicthesis/issues/45/problema-testatine-su-classicthesis-style
% ********************************************************************
%\linespread{1.05} % a bit more for Palatino
% ****************************************************************************************************
